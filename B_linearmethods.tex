\section{Linear Inverse Method}

The goal of the inverse estimation is to simultaneously estimate a) absolute (Earth-referenced) ocean velocity profile, and b) the glider path through the water, based on on-board ADCP observations of relative current velocity profiles.
For simplicity, the derivation is in terms of scalar velocity $v$, which can be understood to be either one of the velocity components $(u,v)$, or a complex-number representation of velocity, $u+iv$.

\subsection {Formulation of the inverse problem}
Similar to \cite{Visbeck2002,todd-2011-JGRC}, glider-borne ADCP observations of horizontal currents are considered to be a sum of three unknown components: 
$$u^a(z,t)=u^o(z)-u^g(t)+u^n(z,t),$$
where 
\begin{itemize}
\item $u^o$ is the absolute ocean velocity 
\item $u^g$ is the over-the-ground (OTG) velocity of the glider platform 
\item $u^n$ is the ADCP measurement noise. 
\end{itemize}
Here, we separate the OTG glider velocity into the drift velocity $u^d=u^o(z^g(t))$, equal to the ocean velocity at the glider depth $z^g$, and the glider horizontal TTW "propulsion" speed $u^p$, which is determined by the glider control and its flight dynamics:
\[
u^g(t) = u^o(z^g(t)) + u^p(t).
\]
Thus the inverse problem is to obtain $u^o(z)$ and $u^g(t)$ such that the discrepancy with the observed values of $u^a$ are minimized,
\begin{equation}
\label{eq:min}
\min{\|u^a(z,t)-(u^o(z)-u^o(z^g(t))-u^p(t))\|_2},
\end{equation}
in this formulation in a least squares sense, but other minimization criteria can be used.

This formulation may appear unnecessarily complex compared to \cite{todd-2011-JGRC}, as it requires interpolation of ocean velocity onto the glider location ($u^o(z^g(t))$), but it allows explicit treatment of the glider control (discussed in Section \ref{sec:nonlin}). Additionally, we feel that application of the regularizing smoothness constraint (described below) to $u^p$ is more appropriate than to $u^g$.

\subsection{Discrete formulation}
We consider a set of individual ADCP observations $u^a_{ij}=u^a(z_{ij},t_j)$ at depth cells $z_{ij}$ and times $t_j$, with the
 $i\in [1,I]$ being the ADCP range cell index, and $j\in [1,J]$ being the sample time (ping) index.  All the valid observations form an observation column vector 
$$u^a=[u^a_1...u^a_k]^\top,$$
where the index $k\in [1, K], K\le IJ$ enumerates valid observations. The vectors $[z_k]$ and $[t_k]$ represent the corresponding depth and time coordinates of the $k$-th valid observation.


The unknown state vector 
$$x=\begin{bmatrix}
{u^g} \\ u^o
\end{bmatrix}$$
consists of a vector of the unknown glider OTG velocities 
$${u^g}= [u^g_1 ... u^g_M]^\top$$ 
defined on a temporal grid $\hat{t}_m, m\in[1,M]$, and a vector of the unknown ocean velocities 
$$u^o=[u^o_1 ... u^o_L]^\top$$
defined on a regular vertical grid $\hat{z}_l=(l-1)\Delta \hat{z}, l\in [1,L]$. For convenience, the temporal grid $\hat{t}_m$ is taken to be a superset of the sample times $t_j$, with the extra intervals added during the gaps in the ADCP record (at the beginning and the end of the dive, as well as during a brief intermission at the bottom of the dive); the spacing of the extra intervals is set to the average ADCP sampling interval.
Discrete formulation of the ADCP sampling problem \eqref{eq:min} then becomes 
$$
u^a= - H^t u^g + H^z u^0 + n, %\mathrm{Hx}+
$$
where $n$ is the noise vector.
%The observation matrix $H$ can be written as 
%$$
%H= \begin{bmatrix}
%&-H^t&|&H^{z} &\end{bmatrix},
%$$
The matrix operator $H^t$ represents temporal interpolation from the time grid $\{\hat{t}_m\}$ onto the sampling times $\{t_k\}$, and 
$H^{z}$ represents spatial interpolation from the vertical grid $\{\hat{z}_l\}$ onto $\{z_k\}$.
Since $\{t_k\}\subset \{\hat{t}_m\}$ by construction, $H^t$ is simply a subsampling matrix,
$$
H^t_{km}=\begin{cases} 1,&\text{if }t_k = \hat{t}_m\\0,&\text{otherwise } \end{cases}.
$$
The formulation of vertical interpolation matrices depends on the chosen interpolation method. Visbeck \cite{Visbeck2002} method is equivalent to the nearest-neighbor interpolation. Here, we employ linear interpolation, corresponding to a matrix operator
$$
H^z_{kl}=\begin{cases}
1-(\hat{z}_l-z_k)/\Delta \hat{z},&\hat{z}_{l-1} \le z_k <\hat{z}_l\\
1-(z_k-\hat{z}_l)/\Delta \hat{z},&\hat{z}_l \le z_k <\hat{z}_{l+1}\\
0,& \text{otherwise.}
\end{cases}
$$

\subsection {Additional constraints}
\label{sec:inverse.constraint}

At the core of the inverse problem is the requirement to balance various
constraints on the glider's motion in the form of additional measurements (such
as start- and end-of-dive GPS fixes) and assumptions we can make about the
physical environment, such as currents vary smoothly in time and space.


\textbf{Start- and End-of-Dive GPS:} GPS fixes before and after the dive provide tie-points on the integral $\int u^gdt$ over the duration of the dive. The discrete formulation of these relationships is given by 
$$
Sx\approx s,
$$
where $s$ is the scalar (complex) horizontal displacement between the two GPS fixes,
\(
S=\begin{bmatrix}w & 0\end{bmatrix},
\)
and $w$ represents the weights corresponding to trapezoidal rule integration, 
$$ w_m=\begin{cases}
0.5(\hat{t}_2-\hat{t}_1), &m=1\\
0.5(\hat{t}_{m+1}-\hat{t}_{m-1}), &1<m<M\\
0.5(\hat{t}_M-\hat{t}_{M-1}), &m=M
\end{cases}
$$

\textbf{Glider Velocity:} A measure of the glider's TTW velocity, such as that provided by the glider hydrodynamic model, can be used to provide a constraint on $u^p$(t) in \eqref{eq:min}. Discrete formulation of this constraint is 
\[
u^p = -H^t u^g + H^{z0} u^o + n, 
\]
%$$\begin{bmatrix}
%&-H^t&|&H^{z0} &\end{bmatrix} x=u^{model},$$
where $u^{p}$ is the TTW velocity  predicted by the hydrodynamic model, and $H^{z0}$ represents spatial interpolation of ocean velocity profile onto the glider depth 
(so that  $H^{z0}  x$ is the glider drift speed); this matrix is constructed in the same way as $H^{z}$. The relative confidence, within the model, of these measurements compared to the ADCP measurements can be controlled by adjusting the relative magnitude of the noise vectors $n$ for the two measurements. As is demonstrated and discussed in the Results section, however, understanding the subtleties of how to weight the constraints is still a work in progress.

\textbf{Smoothness Regularization:}
\label{sec:inv.reg}
Additionally, we require both the ocean and the glider velocities to be smooth, which is equivalent to minimization of ${D_2u^g}$, and ${D_2u^o}$ where $D_2$ are the second derivative operators of the appropriate sizes,
$$D_2=\begin{bmatrix}
1 &-2 &1 &0 &\cdots &0\\
0 &1 &-2 &1 &\\
\vdots & & & \ddots &\ddots\\
0 &\cdots &0 &1 &-2 &1
\end{bmatrix}.
$$


\textbf{Subtleties of Regularization:}
%
Ideally, we want to solve an inverse problem that is aware of both the
vehicle dynamics and control and the ADCP observations, because this is the
only way to make a self-consistent estimate. The challenge, as mentioned above
with the glider velocity estimates in particular, is that measurements are inherently
noisy, often biased, and optimally choosing the relative weighting between
measurements is still a work in progress.
%
%We have some idea how  the glider moves, and that knowledge should be useful.
%

In the extreme case, if we had a perfect hydrodynamic model, we would not need
the inverse at all, as we would know the TTW velocity at any moment (based
solely on the buoyancy and pitch control), and therefore would have absolute
ocean velocity estimates from every trace. At the other extreme, if the ADCP
measurements were perfect and extended close to the vehicle, we would have a
direct measure of the TTW velocity from that information alone and would not
need the dynamic model.

In reality, the ADCP is noisy and has gaps (notably, the blanking
distance). And without the hydrodynamic model, we can't distinguish between the legitimate accelerations due to AUG control (which
should be kept) and spurious accelerations arising from ADCP noise (which should
be eliminated)---and we would be forced to either keep both, or eliminate both,
which is suboptimal.

%so some smoothing and extrapolation is required to estimate the TTW
%velocity from ADCP. Smoothing things uniformly, however, presents another
%problem---we can expect the glider to move smoothly \emph{except} when it changes
%pitch or buoyancy.

%Including the model TTW velocity estimate constraint in the inverse should help, theoretically, as long as a) it is reasonably correct, and b) we have a good idea of what is considered "allowable" deviation from it -- because we need to tell the inverse what is considered to be "close" to the model TTW velocity (i.e., misfit measure). 

%Right now, KF has both the long-term (bias) and short-term (oversmoothing) issues, which are difficult to control simultaneously. (If it were only noisy, we could require the solution to be close to KF on average; if it were only biased, we could require the deviation to be smooth. Cannot efficiently require both.) 

%%%%%%%%%%%%%%

\subsection {Least-Squares Formulation}
The final least-squares problem formulation is given by 
$$
\min_x{\|{Gx-d}\|_2^2},
$$
where
$$
G=
\begin{bmatrix}
-H^t  &H^z \\
w & 0\\
 -H^t & H^{z0}\\
r_o D_2 & 0 \\
0  & r_gD_2
\end{bmatrix}, \qquad 
d=\begin{bmatrix}{u^a}\\ s\\  u^p \\ 0 \\ 0  \end{bmatrix},
$$
and $r_o$ and $r_g$ are regularization parameters.
The solution is given by
$$
x=(G^\top G)^{-1}G^\top d,
$$
and can be computed in a more efficient manner (e.g., using a QR decomposition aware of the block structure of $G$).
\subsection {Modification: two-profile solution}
The ocean velocity profile is expected to change over the duration of the dive. Therefore, it may be reasonable to seek \emph{two} ocean velocity profiles $u^d$ and $u^u$, corresponding to the descent and asent. The state vector and the observation matrix then become
$$x=\begin{bmatrix}
{u^g} \\ 
u^d \\
u^u 
\end{bmatrix},$$
$$
H= \begin{bmatrix}
&-H^t&|&H^{zd} &|&H^{zu} &\end{bmatrix},
$$
where the interpolation matrices $H^{zd}$ and $H^{zu}$ are constructed as $H^{z}$ before, except that only those rows of $H^{zd}$ that correspond to the downcast $z_k$ are non-zero, and vise versa. 

An additional constraint is necessary, requiring the ocean velocity profiles to match at the bottom end, i.e.
$$
\begin{bmatrix}
0 & e_n^\top & -e_n^\top
\end{bmatrix}x=0.
$$
where $e_n$ is the elementary vector with $1$ in the last entry. 

The modified system is given by 

$$
\min_x{\|{Gx-d}\|_2^2},
$$
where
$$
G=
\begin{bmatrix}
-H^t  &H^{zd} & H^{zu} \\
w & 0 & 0\\
 -H^t & H^{z0d} &H^{z0u} \\
r_o D_2 & 0 & 0\\
0  & r_gD_2 & r_gD_2 
\end{bmatrix}, \qquad 
d=\begin{bmatrix}{u^a}\\ s\\  u^p \\ 0 \\ 0  \end{bmatrix},
$$



For shorter dives, assuming identical current profiles on ascent and descent may be a reasonable assumptions. For the data collected over multi-hour dives, however, as we show, there can be significant deviation in the current profile during descent and ascent, and we use the two-profile solution.
