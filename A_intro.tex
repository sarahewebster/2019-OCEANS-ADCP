\section{Introduction \textbf(SARAH)}

The main challenge of AUG-based ADCP velocity profiling is that each velocity profile is measured relative to the TTW motion of the glider, which is generally unknown. TTW velocity can be inferred from the AUG dynamic model, but is subject to uncertainty because the model is based on steady state flight and doesn't take roll into account. This lack of a georeferenced platform velocity produces the need to perform a simultaneous robust estimation of current and glider TTW velocity using all available data -- ADCP velocity profiles, surrounding water density, glider buoyancy engine state, glider attitude and angle of attack, glider depth, and GPS positions at the start and end of the dive.

\section{Background \textbf(SARAH)}

Extensive work by Todd et al in this domain employed a modified version of the lowered ADCP estimation framework \cite{todd-2017-JAOT,todd-2011-JGRC}.  In addition to this estimation framework, our work uses current data from the entire dive (as opposed to just during descent) and incorporates nonlinear hydrodynamic and attitude models from the vehicle navigation system to produce a time-varying current profile for the entire dive.

SARAH -- complete literature review.