%% bare_conf.tex
%% V1.4
%% 2012/12/27
%% by Michael Shell
%% See:
%% http://www.michaelshell.org/
%% for current contact information.
%%
%% This is a skeleton file demonstrating the use of IEEEtran.cls
%% (requires IEEEtran.cls version 1.8 or later) with an IEEE conference paper.
%%
%% Support sites:
%% http://www.michaelshell.org/tex/ieeetran/
%% http://www.ctan.org/tex-archive/macros/latex/contrib/IEEEtran/
%% and
%% http://www.ieee.org/

%%*************************************************************************
%% Legal Notice:
%% This code is offered as-is without any warranty either expressed or
%% implied; without even the implied warranty of MERCHANTABILITY or
%% FITNESS FOR A PARTICULAR PURPOSE! 
%% User assumes all risk.
%% In no event shall IEEE or any contributor to this code be liable for
%% any damages or losses, including, but not limited to, incidental,
%% consequential, or any other damages, resulting from the use or misuse
%% of any information contained here.
%%
%% All comments are the opinions of their respective authors and are not
%% necessarily endorsed by the IEEE.
%%
%% This work is distributed under the LaTeX Project Public License (LPPL)
%% ( http://www.latex-project.org/ ) version 1.3, and may be freely used,
%% distributed and modified. A copy of the LPPL, version 1.3, is included
%% in the base LaTeX documentation of all distributions of LaTeX released
%% 2003/12/01 or later.
%% Retain all contribution notices and credits.
%% ** Modified files should be clearly indicated as such, including  **
%% ** renaming them and changing author support contact information. **
%%
%% File list of work: IEEEtran.cls, IEEEtran_HOWTO.pdf, bare_adv.tex,
%%                    bare_conf.tex, bare_jrnl.tex, bare_jrnl_compsoc.tex,
%%                    bare_jrnl_transmag.tex
%%*************************************************************************

% *** Authors should verify (and, if needed, correct) their LaTeX system  ***
% *** with the testflow diagnostic prior to trusting their LaTeX platform ***
% *** with production work. IEEE's font choices can trigger bugs that do  ***
% *** not appear when using other class files.                            ***
% The testflow support page is at:
% http://www.michaelshell.org/tex/testflow/



% Note that the a4paper option is mainly intended so that authors in
% countries using A4 can easily print to A4 and see how their papers will
% look in print - the typesetting of the document will not typically be
% affected with changes in paper size (but the bottom and side margins will).
% Use the testflow package mentioned above to verify correct handling of
% both paper sizes by the user's LaTeX system.
%
% Also note that the "draftcls" or "draftclsnofoot", not "draft", option
% should be used if it is desired that the figures are to be displayed in
% draft mode.
%
\documentclass[conference]{IEEEtran}
% Add the compsoc option for Computer Society conferences.
%
% If IEEEtran.cls has not been installed into the LaTeX system files,
% manually specify the path to it like:
% \documentclass[conference]{../sty/IEEEtran}





% Some very useful LaTeX packages include:
% (uncomment the ones you want to load)


% *** MISC UTILITY PACKAGES ***
%
%\usepackage{ifpdf}
% Heiko Oberdiek's ifpdf.sty is very useful if you need conditional
% compilation based on whether the output is pdf or dvi.
% usage:
% \ifpdf
%   % pdf code
% \else
%   % dvi code
% \fi
% The latest version of ifpdf.sty can be obtained from:
% http://www.ctan.org/tex-archive/macros/latex/contrib/oberdiek/
% Also, note that IEEEtran.cls V1.7 and later provides a builtin
% \ifCLASSINFOpdf conditional that works the same way.
% When switching from latex to pdflatex and vice-versa, the compiler may
% have to be run twice to clear warning/error messages.






% *** CITATION PACKAGES ***
%
%\usepackage{cite}
% cite.sty was written by Donald Arseneau
% V1.6 and later of IEEEtran pre-defines the format of the cite.sty package
% \cite{} output to follow that of IEEE. Loading the cite package will
% result in citation numbers being automatically sorted and properly
% "compressed/ranged". e.g., [1], [9], [2], [7], [5], [6] without using
% cite.sty will become [1], [2], [5]--[7], [9] using cite.sty. cite.sty's
% \cite will automatically add leading space, if needed. Use cite.sty's
% noadjust option (cite.sty V3.8 and later) if you want to turn this off
% such as if a citation ever needs to be enclosed in parenthesis.
% cite.sty is already installed on most LaTeX systems. Be sure and use
% version 4.0 (2003-05-27) and later if using hyperref.sty. cite.sty does
% not currently provide for hyperlinked citations.
% The latest version can be obtained at:
% http://www.ctan.org/tex-archive/macros/latex/contrib/cite/
% The documentation is contained in the cite.sty file itself.






% *** GRAPHICS RELATED PACKAGES ***
%
\ifCLASSINFOpdf
  % \usepackage[pdftex]{graphicx}
  % declare the path(s) where your graphic files are
  % \graphicspath{{../pdf/}{../jpeg/}}
  % and their extensions so you won't have to specify these with
  % every instance of \includegraphics
  % \DeclareGraphicsExtensions{.pdf,.jpeg,.png}
\else
  % or other class option (dvipsone, dvipdf, if not using dvips). graphicx
  % will default to the driver specified in the system graphics.cfg if no
  % driver is specified.
  % \usepackage[dvips]{graphicx}
  % declare the path(s) where your graphic files are
  % \graphicspath{{../eps/}}
  % and their extensions so you won't have to specify these with
  % every instance of \includegraphics
  % \DeclareGraphicsExtensions{.eps}
\fi
% graphicx was written by David Carlisle and Sebastian Rahtz. It is
% required if you want graphics, photos, etc. graphicx.sty is already
% installed on most LaTeX systems. The latest version and documentation
% can be obtained at: 
% http://www.ctan.org/tex-archive/macros/latex/required/graphics/
% Another good source of documentation is "Using Imported Graphics in
% LaTeX2e" by Keith Reckdahl which can be found at:
% http://www.ctan.org/tex-archive/info/epslatex/
%
% latex, and pdflatex in dvi mode, support graphics in encapsulated
% postscript (.eps) format. pdflatex in pdf mode supports graphics
% in .pdf, .jpeg, .png and .mps (metapost) formats. Users should ensure
% that all non-photo figures use a vector format (.eps, .pdf, .mps) and
% not a bitmapped formats (.jpeg, .png). IEEE frowns on bitmapped formats
% which can result in "jaggedy"/blurry rendering of lines and letters as
% well as large increases in file sizes.
%
% You can find documentation about the pdfTeX application at:
% http://www.tug.org/applications/pdftex





% *** MATH PACKAGES ***
%
\usepackage[cmex10]{amsmath}
% A popular package from the American Mathematical Society that provides
% many useful and powerful commands for dealing with mathematics. If using
% it, be sure to load this package with the cmex10 option to ensure that
% only type 1 fonts will utilized at all point sizes. Without this option,
% it is possible that some math symbols, particularly those within
% footnotes, will be rendered in bitmap form which will result in a
% document that can not be IEEE Xplore compliant!
%
% Also, note that the amsmath package sets \interdisplaylinepenalty to 10000
% thus preventing page breaks from occurring within multiline equations. Use:
%\interdisplaylinepenalty=2500
% after loading amsmath to restore such page breaks as IEEEtran.cls normally
% does. amsmath.sty is already installed on most LaTeX systems. The latest
% version and documentation can be obtained at:
% http://www.ctan.org/tex-archive/macros/latex/required/amslatex/math/





% *** SPECIALIZED LIST PACKAGES ***
%
%\usepackage{algorithmic}
% algorithmic.sty was written by Peter Williams and Rogerio Brito.
% This package provides an algorithmic environment fo describing algorithms.
% You can use the algorithmic environment in-text or within a figure
% environment to provide for a floating algorithm. Do NOT use the algorithm
% floating environment provided by algorithm.sty (by the same authors) or
% algorithm2e.sty (by Christophe Fiorio) as IEEE does not use dedicated
% algorithm float types and packages that provide these will not provide
% correct IEEE style captions. The latest version and documentation of
% algorithmic.sty can be obtained at:
% http://www.ctan.org/tex-archive/macros/latex/contrib/algorithms/
% There is also a support site at:
% http://algorithms.berlios.de/index.html
% Also of interest may be the (relatively newer and more customizable)
% algorithmicx.sty package by Szasz Janos:
% http://www.ctan.org/tex-archive/macros/latex/contrib/algorithmicx/




% *** ALIGNMENT PACKAGES ***
%
%\usepackage{array}
% Frank Mittelbach's and David Carlisle's array.sty patches and improves
% the standard LaTeX2e array and tabular environments to provide better
% appearance and additional user controls. As the default LaTeX2e table
% generation code is lacking to the point of almost being broken with
% respect to the quality of the end results, all users are strongly
% advised to use an enhanced (at the very least that provided by array.sty)
% set of table tools. array.sty is already installed on most systems. The
% latest version and documentation can be obtained at:
% http://www.ctan.org/tex-archive/macros/latex/required/tools/


% IEEEtran contains the IEEEeqnarray family of commands that can be used to
% generate multiline equations as well as matrices, tables, etc., of high
% quality.




% *** SUBFIGURE PACKAGES ***
%\ifCLASSOPTIONcompsoc
%  \usepackage[caption=false,font=normalsize,labelfont=sf,textfont=sf]{subfig}
%\else
%  \usepackage[caption=false,font=footnotesize]{subfig}
%\fi
% subfig.sty, written by Steven Douglas Cochran, is the modern replacement
% for subfigure.sty, the latter of which is no longer maintained and is
% incompatible with some LaTeX packages including fixltx2e. However,
% subfig.sty requires and automatically loads Axel Sommerfeldt's caption.sty
% which will override IEEEtran.cls' handling of captions and this will result
% in non-IEEE style figure/table captions. To prevent this problem, be sure
% and invoke subfig.sty's "caption=false" package option (available since
% subfig.sty version 1.3, 2005/06/28) as this is will preserve IEEEtran.cls
% handling of captions.
% Note that the Computer Society format requires a larger sans serif font
% than the serif footnote size font used in traditional IEEE formatting
% and thus the need to invoke different subfig.sty package options depending
% on whether compsoc mode has been enabled.
%
% The latest version and documentation of subfig.sty can be obtained at:
% http://www.ctan.org/tex-archive/macros/latex/contrib/subfig/




% *** FLOAT PACKAGES ***
%
%\usepackage{fixltx2e}
% fixltx2e, the successor to the earlier fix2col.sty, was written by
% Frank Mittelbach and David Carlisle. This package corrects a few problems
% in the LaTeX2e kernel, the most notable of which is that in current
% LaTeX2e releases, the ordering of single and double column floats is not
% guaranteed to be preserved. Thus, an unpatched LaTeX2e can allow a
% single column figure to be placed prior to an earlier double column
% figure. The latest version and documentation can be found at:
% http://www.ctan.org/tex-archive/macros/latex/base/


%\usepackage{stfloats}
% stfloats.sty was written by Sigitas Tolusis. This package gives LaTeX2e
% the ability to do double column floats at the bottom of the page as well
% as the top. (e.g., "\begin{figure*}[!b]" is not normally possible in
% LaTeX2e). It also provides a command:
%\fnbelowfloat
% to enable the placement of footnotes below bottom floats (the standard
% LaTeX2e kernel puts them above bottom floats). This is an invasive package
% which rewrites many portions of the LaTeX2e float routines. It may not work
% with other packages that modify the LaTeX2e float routines. The latest
% version and documentation can be obtained at:
% http://www.ctan.org/tex-archive/macros/latex/contrib/sttools/
% Do not use the stfloats baselinefloat ability as IEEE does not allow
% \baselineskip to stretch. Authors submitting work to the IEEE should note
% that IEEE rarely uses double column equations and that authors should try
% to avoid such use. Do not be tempted to use the cuted.sty or midfloat.sty
% packages (also by Sigitas Tolusis) as IEEE does not format its papers in
% such ways.
% Do not attempt to use stfloats with fixltx2e as they are incompatible.
% Instead, use Morten Hogholm'a dblfloatfix which combines the features
% of both fixltx2e and stfloats:
%
% \usepackage{dblfloatfix}
% The latest version can be found at:
% http://www.ctan.org/tex-archive/macros/latex/contrib/dblfloatfix/




% *** PDF, URL AND HYPERLINK PACKAGES ***
%
%\usepackage{url}
% url.sty was written by Donald Arseneau. It provides better support for
% handling and breaking URLs. url.sty is already installed on most LaTeX
% systems. The latest version and documentation can be obtained at:
% http://www.ctan.org/tex-archive/macros/latex/contrib/url/
% Basically, \url{my_url_here}.




% *** Do not adjust lengths that control margins, column widths, etc. ***
% *** Do not use packages that alter fonts (such as pslatex).         ***
% There should be no need to do such things with IEEEtran.cls V1.6 and later.
% (Unless specifically asked to do so by the journal or conference you plan
% to submit to, of course. )


% correct bad hyphenation here
\hyphenation{op-tical net-works semi-conduc-tor}

\usepackage{graphicx}
\usepackage{acronym}
\usepackage{xspace}


\begin{document}


%
% paper title
% can use linebreaks \\ within to get better formatting as desired
% Do not put math or special symbols in the title.
\title{Preliminary Results in ADCP-aided Navigation for Autonomous Underwater Gliders}
%
% author names and affiliations
% use a multiple column layout for up to three different
% affiliations
%\author{
%Michael V. Jakuba$^{1}$, 
%James C. Kinsey$^{1}$, 
%James Partan$^{1}$,
%\& Sarah E. Webster$^{2}$
%% Lee Freitag$^{1}$,
%% Adam Soule$^{1}$, \&
%% Christopher R. German$^{1}$,
%% <-this % stops a space
%\thanks{$^{1}$Woods Hole Oceanographic Engineering, Woods Hole, MA USA
%        {\tt\small @whoi.edu}}%
%\thanks{$^{2}$Applied Physics Laboratory, University of Washington, Seattle, WA USA}%
%}


% conference papers do not typically use \thanks and this command
% is locked out in conference mode. If really needed, such as for
% the acknowledgment of grants, issue a \IEEEoverridecommandlockouts
% after \documentclass

% for over three affiliations, or if they all won't fit within the width
% of the page, use this alternative format:
% 
\author{\IEEEauthorblockN{Sarah E. Webster\IEEEauthorrefmark{1},
Andrey Shcherbina\IEEEauthorrefmark{1} and
Aleksandr Aravkin\IEEEauthorrefmark{2}}
\IEEEauthorblockA{\IEEEauthorrefmark{1}Applied Physics Laboratory, University of Washington, Seattle, WA USA}
\IEEEauthorblockA{\IEEEauthorrefmark{2}Applied Mathematics, University of Washington, Seattle, WA USA}
}




% use for special paper notices
%\IEEEspecialpapernotice{(Invited Paper)}

% make the title area
\maketitle


\vspace{-0.2in}

% my_acronyms.sty
%
% define acronyms used with my thesis

% #'s
\acrodef{2D}{two-dimensional}
\acrodef{3D}{three-dimensional}

% A
\acrodef{ABE}{Autonomous Benthic Explorer}
\acrodef{ADCP}{acoustic Doppler current profiler}
\acrodef{ADV}{acoustic Doppler velocimeter}
\acrodef{Alvin}{Alvin}
\acrodef{AFRL}{Air Force Research Laboratory}
\acrodef{AHRS}{attitude and heading reference system}
\acrodef{Autosub}{Autosub}
\acrodef{AUG}{autonomous underwater glider}
\acrodef{AUV}{autonomous underwater vehicle}
\acrodef{AGU}{American Geophysical Union}
\acrodef{AON}{Arctic Observing Network}
\acrodef{AOPE}{Applied Ocean Physics and Engineering}
\acrodef{AS}{asymptotically stable}
\acrodef{ACFR}{Australian Centre for Field Robotics}
\acrodef{ASV}{autonomous surface vehicle}

% B
\acrodef{BCC}{brightness constancy constraint}
\acrodef{BIO}{Bedford Institute of Oceanography}

% C
\acrodef{CAD}{computer aided design}
\acrodef{CenSSIS}{Center for Subsurface Sensing and Imaging Systems}
\acrodef{CCD}{charge coupled device}
\acrodef{CI}{covariance intersection}
\acrodef{CG}{conjugate gradients}
\acrodef{COG}{course over ground}
\acrodef{CPU}{central processing unit}
\acrodef{CT}{continuous-time}
\acrodef{CB}{center of buoyancy}
\acrodef{CG}{center of gravity}
\acrodef{CGSN}{Canadian Gravity Standardization Net}
\acrodef{COM}{center of mass}
\acrodef{CO2}{carbon dioxide}
\acrodef{CPS}{control power supply}
\acrodef{CSAC}{chip-scale atomic clock}
\acrodef{CST}{{\it IEEE Transactions on Control Systems Technology}}
\acrodef{CSV}{Comma Separated Values}
\acrodef{CTD}{conductivity-temperature-depth}
\acrodef{CISE}{Computer and Information Science and Engineering}

% D
\acrodef{DOF}{degree of freedom}
\acrodef{DoD}{Department of Defense}
\acrodef{DSL}{Deep Submergence Laboratory}
\acrodef{DT}{discrete-time}
\acrodef{DVL}{Doppler velocity log}
\acrodef{DPM}{digital panel meter}
\acrodef{DR}{dead reckoning}
\acrodef{DRI}{department research initiative}
\acrodef{DWH}{Deepwater Horizon}

% E
\acrodef{ESDF}{exactly sparse delayed-state filter}
\acrodef{EM}{electromagnetic}
\acrodef{EKF}{extended Kalman filter}
\acrodef{EIF}{extended information filter}
\acrodef{ERC}{Engineering Research Center}
\acrodef{EPR}{East Pacific Rise}
\acrodef{EPSL}{{\it Earth and Planetary Science Letters}}

% F
\acrodef{FastSLAM}{Factored Solution to SLAM}
\acrodef{FOG}{fiber optic gyro}
\acrodef{FOV}{field of view}
\acrodef{FFT}{fast Fourier transform}
\acrodef{FBD}{free body diagram}
\acrodef{FIRST}{For Inspiration and Recognition of Science and Technology}

% G
\acrodef{GMRF}{Gaussian Markov random field}
\acrodef{GPS}{global positioning system}
\acrodef{GAS}{globally asymptotically stable}
\acrodef{GA}{geometric algebra}
\acrodef{GUI}{graphical user interface}

% H
\acrodef{HUGIN}{HUGIN}
\acrodef{HMMV}{H\r{a}kon Mosby Mud Volcano}
\acrodef{HOV}{human occupied vehicle}
\acrodef{HROV}{hybrid remotely operated vehicle}
\acrodef{HTF}{Hydrodynamics Test Facility}

% I
\acrodef{ICRA}{{\it IEEE International Conference on Robotics and Automation}}
\acrodef{IF}{information filter}
\acrodef{IFREMER}{French Institute for the Research and Exploitation of the Sea}
\acrodef{IFE}{Institute for Exploration}
\acrodef{INU}{inertial navigation unit}
\acrodef{INS}{inertial navigation system}
\acrodef{IR}{infrared}
\acrodef{IMU}{inertial measurement unit}
\acrodef{INS}{inertial navigation system}
\acrodef{IBCAO}{International Bathymetric Chart of the Arctic Ocean}
\acrodef{IROS}{{\it IEEE International Conference on Robotics and Intelligent Systems}}
\acrodef{iUSBL}{inverted USBL}
\acrodef{OWTTiUSBL}{One-Way Travel-Time inverted USBL}

% J
\acrodef{Jason}{Jason}
\acrodef{JFR}{{\it Journal of Field Robotics}}
\acrodef{JdF}{Juan de Fuca}
\acrodef{JHU}{Johns Hopkins University}
\acrodef{JPEG}{Joint Photographic Experts Group}

% K
\acrodef{KF}{Kalman filter}
\acrodef{KYP}{Kalman-Yakubovich-Popov}
\acrodef{KISS}{Keck Institute for Space Studies}

% L
\acrodef{LADCP}{lowered acoustic Doppler current profiler}
\acrodef{LBL}{long-baseline}
\acrodef{LCM}{lightweight communication and marshaling}
\acrodef{LG}{linear Gaussian}
\acrodef{LKY}{Lefschetz-Kalman-Yakubovich}
\acrodef{LMedS}{least median of squares}
\acrodef{LLE}{Linear Lyapunov Equation}
\acrodef{LQR}{Linear Quadratic Regulation}
\acrodef{LQGR}{Linear Quadratic Gaussian Regulation}
\acrodef{LSSL}{Louis S. St Laurent}
\acrodef{LRAUV}{long-range AUV}

% M
\acrodef{MAP}{maximum \emph{a posteriori}}
\acrodef{MBARI}{Monterey Bay Aquarium Research Institute}
\acrodef{MBN}{mosaic-based navigation}
\acrodef{MEF}{Main Endeavor Field}
\acrodef{MIT}{Massachusetts Institute of Technology}
\acrodef{MLE}{maximum likelihood estimate}
\acrodef{MRF}{Markov random field}
\acrodef{MIZ}{Marginal Ice Zone}
\acrodef{MATE}{Marine Advanced Technology Education}
\acrodef{MCR}{Mid-Cayman Rise}
\acrodef{MEMS}{micro-electrical-mechanical systems}
\acrodef{MBARI}{Monterey Bay Aquarium Research Institute}
\acrodef{MIMO}{multiple-input, multiple-output}
\acrodef{MOR}{mid-ocean ridge}
\acrodef{MVCO}{Martha's Vineyard Coastal Observatory}
\acrodef{MCL}{mission critical level}
\acrodef{MO}{mission objective}
\acrodef{MSD}{mass spring damper}
% N
\acrodef{NavEst}{navigation estimation}
\acrodef{NDSEG}{National Defense Science and Engineering Graduate}
\acrodef{NEES}{normalized estimation error squared}
\acrodef{NDSF}{National Deep Submergence Facility}
\acrodef{NMEA}{National Marine Electronics Association}
\acrodef{NSF}{National Science Foundation}
\acrodef{NTP}{network time protocol}
\acrodef{NAS}{National Academy of Sciences}
\acrodef{NDSF}{National Deep Submergence Facility}
\acrodef{NHS}{Natick High School}
\acrodef{NLO}{nonlinear observer}
\acrodef{NSTA}{National Science Teachers Association}
\acrodef{NSIDC}{National Snow and Ice Data Center}
\acrodef{NUI}{Nereid under-ice}
% O
\acrodef{OS}{operating system}
\acrodef{OWTT}{one-way travel time}
\acrodef{ODE}{ordinary differential equation}
\acrodef{OL}{Second-Order, Open-Loop Observer}
\acrodef{ONR}{Office of Naval Research}
\acrodef{OOI}{Ocean Observing Initiative}
\acrodef{OWTT}{one-way travel time}


% P
\acrodef{PC}{personal computer}
\acrodef{PCB}{printed circuit board}
\acrodef{PDE}{partial differential equation}
\acrodef{PPS}{pulse per second}
\acrodef{PEG}{parameter error gain}
\acrodef{PHF}{Piccard Hydrothermal Field}
\acrodef{PNAS}{{\it Proceedings of the National Academy of Sciences}}
\acrodef{ppb}{part-per-billion}
\acrodef{ppm}{part-per-million}
\acrodef{PR}{Positive Real}
\acrodef{PROV}{polar remotely operated vehicle}
\acrodef{PI}{principal investigator}

% Q

% R
\acrodef{RAM}{random access memory}
\acrodef{RANSAC}{random sample consensus}
\acrodef{RDI}{RD Instruments}
\acrodef{REMUS}{Remote Environmental Monitoring Unit}
\acrodef{RLG}{ring laser gyroscope}
\acrodef{RF}{radio frequency}
\acrodef{RMS}{Royal Mail Steamship}
\acrodef{ROM}{range only measurement}
\acrodef{ROV}{remotely operated vehicle}
\acrodef{RTC}{real-time clock}
\acrodef{ROV}{remotely operated vehicle}
\acrodef{ROVER}{remotely operated vehicle environmental research}
\acrodef{RNS}{Regional Scale Node}
\acrodef{RSS}{{\it Robotics: Science and Systems}}
\acrodef{RTS}{Rauch-Tung-Striebel}

% S
\acrodef{SeaBED}{SeaBED}
\acrodef{SEIF}{sparse extended information filter}
\acrodef{SIFT}{scale invariant feature transform}
\acrodef{SLAM}{simultaneous localization and mapping}
\acrodef{SNAME}{The Society of Naval Architects and Marine Engineers}
\acrodef{SSD}{sum of squared differences}
\acrodef{SFM}{structure-from-motion}
\acrodef{SO}{special orthogonal}
\acrodef{SOA}{state of the art}
\acrodef{SPR}{Strictly Positive Real}
\acrodef{STEM}{Science, Technology, Engineering, and Mathematics }
\acrodef{SSF}{Summer Student Fellow}
\acrodef{SVO}{Scaler, Velocity Observer}
\acrodef{SVD}{singular value decomposition}
\acrodef{SVP}{sound velocity profile}
\acrodef{SPURS}{Salinity Processes in the Upper Ocean Regional Study}
% T
\acrodef{TDMA}{time division multiple access}
\acrodef{TRL}{technology readiness level}
\acrodef{TRO}{{\it IEEE Transactions on Robotics}}
\acrodef{TJTF}{thin junction-tree filter}
\acrodef{TTL}{transistor-transistor logic}
\acrodef{TXCO}{temperature compensated crystal oscillator}
\acrodef{3D}{three dimensional}
\acrodef{TWTT}{two-way travel time}

% U
\acrodef{USBL}{ultra-short-baseline}
\acrodef{UUV}{unmanned underwater vehicle}
\acrodef{UTC}{Coordinate Universal Time}
\acrodef{UDP}{User Datagram Protocol}
\acrodef{UKF}{Unscented Kalman Filter}
\acrodef{UV}{Underwater Vehicle}
\acrodef{UNOLS}{University National Oceanographic Laboratory System}
\acrodef{USGS}{United States Geological Survey}
\acrodef{USNA}{United States Naval Academy}
\acrodef{UNCLOS}{United Nations Convention on the Law of the Sea}
% V
\acrodef{VAN}{visually augmented navigation}
\acrodef{VNL}{vision numerical library}
\acrodef{VTK}{The Visualization Toolkit}
\acrodef{VIGA}{vehicle induced gravimeter acceleration}

% W
\acrodef{WHOI}{Woods Hole Oceanographic Institution}

% X

% Y
\acrodef{YIP}{Young Investigator Program}

% Z
%NEW COMMAND LIST
% this is a command for superscript and subscript on the left
%\newcommand{\preind}[3]{\;{{\small{#1}}\atop{\small{#2}}}#3}
\newcommand{\preind}[3]{\;{{\tiny{#1}}\atop{\tiny{#2}}}\hspace{-0.05in}#3}
\newcommand{\laplace}{\mathcal{L}}
\newcommand{\rb}[1]{\raisebox{-1.5ex}[0pt]{#1}}

\newcommand{\sentry}[0]{{\it Sentry }}
\newcommand{\nereus}[0]{{\it Nereus }}
\newcommand{\jason}[0]{{\it Jason }}
\newcommand{\alvin}[0]{{\it Alvin }}
\newcommand{\iver}[0]{{\it Iver2 }}
\newcommand{\tioga}[0]{{\it Tioga }}

\newcommand{\order}[1]{\ensuremath{\mathcal{O}(10^{#1})}}

% Eriksen uses the name ``Deepglider.''
\newcommand{\deepGlider}{Deepglider}

% These names suck.
\newcommand{\standardTraj}{conventional\xspace}
\newcommand{\USBLTraj}{deep-profiling\xspace}

\newcommand{\effFactor}{R}
\newcommand{\effFactorDive}{R_{dive}}


\newcommand{\TDive}{T_{dive}}
\newcommand{\Tconv}{T_{c}} %jck -- time a conventional glider spends submerged
\newcommand{\Tdeep}{T_{d}} %jck -- time a deep profiling glider spends submerged


\newcommand{\profileDepth}{z}
\newcommand{\profileHeight}{\Delta_z}
\newcommand{\diveSpeed}{U}
\newcommand{\vertSpeed}{W}
\newcommand{\diveAngle}{\theta}
\newcommand{\pumpEnergy}{E_{pump}}
\newcommand{\batteryEnergy}{B}
\newcommand{\pHotel}{P_{hotel}}
\newcommand{\pNav}{P_{nav}}
\newcommand{\diveEnergyConventional}{E_{dive,o}}
\newcommand{\diveEnergyUSBL}{E_{dive}}
\newcommand{\xenduranceConventional}{T_o}
\newcommand{\xenduranceUSBL}{T}


%commands for the 2015 RI:small proposal -----
%2015/07/27 21:06:10  changing some notation for OWTT-USBL.


%super- and subscripts to frames and alignments.
\newcommand{\vehicle}[0]{v}
\newcommand{\world}[0]{w}
\newcommand{\usbl}[0]{u}
\newcommand{\locallevel}[0]{n}  % local north-east-down.

% generic position
\newcommand{\pos}[2]{\preind{#1}{}{\mathbf p}_{#2}}

% generic rotation
\newcommand{\rot}[2]{\preind{#1}{#2}\mathbf{R}}

%owtt-usbl
\newcommand{\upu}[0]{\pos{\usbl}{\mathrm{ASV}}}
\newcommand{\wpv}[0]{\pos{\locallevel}{\mathrm{ASV}}}
\newcommand{\vpw}[0]{\pos{\vehicle}{\world}}
\newcommand{\vuR}[0]{\rot{\vehicle}{\usbl}}
\newcommand{\uvR}[0]{\rot{\usbl}{\vehicle}}
\newcommand{\wvR}[0]{\rot{\locallevel}{\vehicle}}
\newcommand{\vwR}[0]{\rot{\vehicle}{\locallevel}}
\newcommand{\az}[0]{\alpha}
\newcommand{\el}[0]{\gamma}
\newcommand{\rng}[0]{\Gamma}

%scalars
\newcommand{\x}[0]{x(t)}
\newcommand{\xdot}[0]{\dot{x}(t)}
\newcommand{\xddot}[0]{\ddot{x}(t)}
\newcommand{\xhat}[0]{\hat{x}(t)}
\newcommand{\xhatdot}[0]{\dot{\hat{x}}(t)}
\newcommand{\xhatddot}[0]{\ddot{\xhat}(t)}
\newcommand{\deltax}[0]{\Delta x(t)}
\newcommand{\deltaxdot}[0]{\Delta \dot{x}(t)}
\newcommand{\deltaxddot}[0]{\Delta \ddot{x}(t)}
\newcommand{\none}[0]{l_1 s_1 + l_2 s_2}
\newcommand{\ntwo}[0]{l_3 s_1 + l_4 s_2 + r}

\newcommand{\vel}[0]{v(t)}
\newcommand{\veldot}[0]{\dot{v}(t)}
\newcommand{\velhat}[0]{\hat{v}(t)}
\newcommand{\velhatdot}[0]{\dot{\hat{v}}(t)}
\newcommand{\veldelta}[0]{\Delta \vel}
\newcommand{\veldeltadot}[0]{\dot{\Delta \vel}}
\newcommand{\veldeltaq}[0]{\Delta v_q(t)}

\newcommand{\out}[0]{w(t)}
\newcommand{\outhat}[0]{\hat{w}(t)}
\newcommand{\outdelta}[0]{\Delta w(t)}

\newcommand{\invec}[0]{\bm{u}(t)}

\newcommand{\measnoise}[0]{n(t)}

%uuv plant parameters with NO indices subscripts
\newcommand{\thrust}[0]{\tau(t)}
\newcommand{\mass}[0]{m}
%% \newcommand{\alpha}[0]{\alpha}
%% \newcommand{\beta}[0]{\beta}
%% \newcommand{\mu}[0]{\mu}
%% \newcommand{\nu}[0]{\nu}
\newcommand{\bouyancy}[0]{b}
\newcommand{\quaddrag}[0]{d_{Q}}
\newcommand{\lindrag}[0]{d_{L}}

%uuv plant parameters with indices subscripts
\newcommand{\veli}[0]{v_i(t)}
\newcommand{\acceli}[0]{\dot{v}_i(t)}
\newcommand{\thrusti}[0]{\tau_i(t)}
\newcommand{\massi}[0]{m_i}
\newcommand{\alphai}[0]{\alpha_i}
\newcommand{\betai}[0]{\beta_i}
\newcommand{\mui}[0]{\mu_i}
\newcommand{\nui}[0]{\nu_i}
\newcommand{\bouyancyi}[0]{b_i}
\newcommand{\quaddragi}[0]{d_{Q_i}}
\newcommand{\lindragi}[0]{d_{L_i}}



%vectors
%\renewcommand{\bm}[1]{{\bm #1}}

%navcal vectors
\newcommand{\worldvel}[0]{ \preind{w}{}\dot{\bm{p}}}
\newcommand{\lblworldvel}[0]{ \preind{w}{}\dot{\bm{p}}_{l}}
\newcommand{\lblworldpos}[0]{ \preind{w}{}\bm{p}_{l}}

\newcommand{\beamvel}[0]{\bm{v}_{beam}(t)}
\newcommand{\dopinstvel}[0]{\preind{i}{}\dot{\bm{p}}_d(t)}
\newcommand{\dopworldvel}[0]{\preind{w}{}\dot{\bm{p}}_{d}(t)}
\newcommand{\dopworldposhat}[0]{\preind{w}{}\hat{\bm{p}}_d(t)}
\newcommand{\dopworldpos}[0]{\preind{w}{}\bm{p}_d(t)}
\newcommand{\dopworldposhatini}[0]{\preind{w}{}\hat{\bm{p}}_{d}(t_0)}
\newcommand{\dopvehvel}[0]{\preind{v}{}\dot{\bm{p}}_d(t)}

%\newcommand{\bvec}[0]{\bm{b}}
\newcommand{\cvec}[0]{\bm{c}}
\newcommand{\qvec}[0]{\bm{q}}
\newcommand{\qvecdot}[0]{\dot{\bm{q}}}


%dynamics and observer vectors
\newcommand{\xvec}[0]{\left[\begin{array}{{c}} \x \\ \xdot \end{array}\right]}
\newcommand{\xvecshort}[0]{{\bm{x}}(t)}
\newcommand{\xvecdot}[0]{\left[\begin{array}{{c}} \xdot \\ \xddot \end{array}\right]}
\newcommand{\xvecdotshort}[0]{\bm{\dot{x}}(t)}


\newcommand{\xhatvec}[0]{\left[\begin{array}{{c}} \xhat \\ \xhatdot \end{array}\right]}
\newcommand{\xhatvecshort}[0]{\bm{\hat{x}}(t)}
\newcommand{\xhatvecdot}[0]{\left[\begin{array}{{c}} \xhatdot \\ \xhatddot \end{array}\right]}
\newcommand{\xhatvecdotshort}[0]{\dot{\bm{\hat{x}}}(t)}

\newcommand{\deltaxvec}[0]{\left[\begin{array}{{c}} \Delta x(t) \\ \dot{\Delta x}(t) \end{array}\right]}
\newcommand{\deltaxvecshort}[0]{\Delta \bm{x}(t)}
\newcommand{\deltaxvecdot}[0]{\left[\begin{array}{{c}} \dot{\Delta x}(t) \\ \ddot{\Delta x}(t) \end{array}\right]}
\newcommand{\deltaxvecdotshort}[0]{\Delta \dot{\bm{x}}(t)}

\newcommand{\betavec}[0]{\left[\begin{array}{{c}} 0 \\ \beta \end{array}\right]}
\newcommand{\betavecshort}[0]{\bm{\beta}}

\newcommand{\alphavec}[0]{\left[\begin{array}{{c}} 0 \\ \alpha \end{array}\right]}
\newcommand{\alphavecshort}[0]{\bm{\alpha}}

\newcommand{\outvec}[0]{\bm{w}(t)}
\newcommand{\outvecshort}[0]{\bm{w}(t)} 
\newcommand{\outhatvecshort}[0]{\bm{\hat{w}}(t)}
\newcommand{\outtildevec}[0]{\bm{\tilde{w}}(t)}
\newcommand{\OUTtilde}[0]{\tilde{W}(t)}
\newcommand{\outdeltavecshort}[0]{\Delta \bm{w}(t)} 

\newcommand{\rvec}[0]{\left[\begin{array}{{c}} 0      \\ r_i(t)   \end{array} \right]}
\newcommand{\svec}[0]{\left[\begin{array}{{c}} s_1(t) \\ s_2(t) \end{array} \right]}
\newcommand{\nuvec}[0]{\left[\begin{array}{{c}} 0 \\ \nu \end{array}\right]}
\newcommand{\nuvecshort}[0]{\bm{\nu}}
\newcommand{\muvecshort}[0]{\bm{\mu}}
\newcommand{\muvec}[0]{\left[\begin{array}{{c}} 0 \\ \mu \end{array}\right]}

\newcommand{\measnoisevec}[0]{\bm{n}(t)}

\newcommand{\instframe}[0]{\preind{i}{}\bm{p}(t)}
\newcommand{\vehicleframe}[0]{\preind{v}{}\bm{p}(t)}
\newcommand{\localframe}[0]{\preind{l}{}\bm{p}(t)}
\newcommand{\worldframe}[0]{\preind{w}{}\bm{p}(t)}

%quadratic scalars
\newcommand{\xq}[0]{\vel |\vel|}
\newcommand{\xqshort}[0]{x_q(t)}
\newcommand{\xhatq}[0]{\dot{\hat{x}}(t) |\dot{\hat{x}}(t)|}
\newcommand{\xhatqshort}[0]{\hat{x}_q(t)}
\newcommand{\deltaxq}[0]{\dot{\hat{x}}(t) |\dot{\hat{x}}(t)| - \dot{x}(t) |\dot{x}(t)|}
\newcommand{\deltaxqshort}[0]{\Delta \dot{x}_q(t)}

%matrices
\newcommand{\Amatrix}[0]{\left[\begin{array}{{cc}} 0 & 1 \\ 0 & \mu \end{array}\right]}
\newcommand{\Ashort}[0]{\bm{A}}
\newcommand{\Cshort}[0]{\bm{C}}
\newcommand{\Pshort}[0]{\bm{P}}
\newcommand{\Lshort}[0]{\bm{L}}
\newcommand{\Qshort}[0]{\bm{Q}}
\newcommand{\eye}[0]{ I}
\newcommand{\momega}[0]{{\bm{m}}(i\omega)}
\newcommand{\ms}[0]{{\bm{m}}(s)}
\newcommand{\inmap}[0]{{\bm{b}}}
\newcommand{\outmap}[0]{\bm{c}}
\newcommand{\Rshort}[0]{\bm{R}}
\newcommand{\insttovehicle}[0]{\preind{v}{i}R}
\newcommand{\vehicletolocal}[0]{\preind{l}{v}R(t)}
\newcommand{\localtoworld}[0]{\preind{w}{l}R(t)}
\newcommand{\vehicletoworld}[0]{\preind{v}{w}R}

\newcommand{\cvecTtrans}[0]{\cvec\hspace{0.03in}^T}
\newcommand{\cvecT}[0]{\cvec}
\newcommand{\cvectrans}[0]{\cvec\hspace{0.03in}^T}
\newcommand{\bvecTtrans}[0]{\bvec\hspace{0.03in}^T}
\newcommand{\bvectrans}[0]{\bvec\hspace{0.03in}^T}
\newcommand{\bvecT}[0]{\bvec}
\newcommand{\qvectrans}[0]{\qvec\hspace{0.03in}^T}
\newcommand{\AT}[0]{A}
\newcommand{\QT}[0]{Q}
\newcommand{\PT}[0]{P}

\newcommand{\zetavec}[0]{\bm{\zeta}}

%define the gravimeter position
\newcommand{\gravposvec}[0]{\preind{w}{}\bm{p_g}}
\newcommand{\gravposx}[0]{\preind{w}{}p_{g_x}}
\newcommand{\gravposy}[0]{\preind{w}{}p_{g_y}}
\newcommand{\gravposz}[0]{\preind{w}{}p_{g_z}}
\newcommand{\gravposveclong}[0]{\left[ \begin{array}{c} \gravposx \\  \gravposy \\ \gravposz \end{array} \right]}
\newcommand{\gravposvecdot}[0]{\preind{w}{}\bm{\dot{p}_g}}
\newcommand{\gravposxdot}[0]{\preind{w}{}\dot{p}_{g_x}}
\newcommand{\gravposydot}[0]{\preind{w}{}\dot{p}_{g_y}}
\newcommand{\gravposzdot}[0]{\preind{w}{}\dot{p}_{g_z}}
\newcommand{\gravposvecddot}[0]{\preind{w}{}\bm{\ddot{p}_g}}
\newcommand{\gravposxddot}[0]{\preind{w}{}\ddot{p}_{g_x}}
\newcommand{\gravposyddot}[0]{\preind{w}{}\ddot{p}_{g_y}}
\newcommand{\gravposzddot}[0]{\preind{w}{}\ddot{p}_{g_z}}

%define the translational offset from the vehicle frame to the gravimeter
\newcommand{\gravoffsetvec}[0]{\preind{g}{v}\bm{d}}
\newcommand{\gravoffsetx}[0]{\preind{g}{v}d_x}
\newcommand{\gravoffsety}[0]{\preind{g}{v}d_y}
\newcommand{\gravoffsetz}[0]{\preind{g}{v}d_z}
\newcommand{\gravoffsetveclong}[0]{\left[ \begin{array}{c} \gravoffsetx \\  \gravoffsety \\ \gravoffsetz \end{array} \right]}


\newcommand{\Rvehtograv}[0]{\preind{g}{v}R}
%define vehicle heading, pitch, and roll
\newcommand{\hdg}[0]{\phi}
\newcommand{\pitch}[0]{\theta}
\newcommand{\roll}[0]{\psi}
\newcommand{\hdgdot}[0]{\dot{\phi}}
\newcommand{\pitchdot}[0]{\dot{\theta}}
\newcommand{\rolldot}[0]{\dot{\psi}}
\newcommand{\hdgddot}[0]{\ddot{\phi}}
\newcommand{\pitchddot}[0]{\ddot{\theta}}
\newcommand{\rollddot}[0]{\ddot{\psi}}

%define the vehicle position
\newcommand{\vehicleposvec}[0]{\preind{w}{}\bm{p_v}}
\newcommand{\vehicleposx}[0]{\preind{w}{}p_{v_x}}
\newcommand{\vehicleposy}[0]{\preind{w}{}p_{v_y}}
\newcommand{\vehicleposz}[0]{\preind{w}{}p_{v_z}}
\newcommand{\vehicleposveclong}[0]{\left[ \begin{array}{c} \vehicleposx \\  \vehicleposy \\ \vehicleposz \end{array} \right]}
\newcommand{\vehicleposxdot}[0]{\preind{w}{}\dot{p}_{v_x}}
\newcommand{\vehicleposydot}[0]{\preind{w}{}\dot{p}_{v_y}}
\newcommand{\vehicleposzdot}[0]{\preind{w}{}\dot{p}_{v_z}}
\newcommand{\vehicleposxddot}[0]{\preind{w}{}\ddot{p}_{v_x}}
\newcommand{\vehicleposyddot}[0]{\preind{w}{}\ddot{p}_{v_y}}
\newcommand{\vehicleposzddot}[0]{\preind{w}{}\ddot{p}_{v_z}}

%define the vehicle velocity vec
\newcommand{\vehiclevelvec}[0]{\left[ \begin{array}{c} \vehicleposxdot \\  \vehicleposydot \\  \vehicleposzdot \\ \hdgdot \\ \pitchdot \\ \rolldot \end{array} \right]}
%define the vehicle accel vec
\newcommand{\vehicleaccvec}[0]{\left[ \begin{array}{c} \vehicleposxddot \\  \vehicleposyddot \\  \vehicleposzddot \\ \hdgddot \\ \pitchddot \\ \rollddot \end{array} \right]}

% As a general rule, do not put math, special symbols or citations in the abstract
%

\newpage

This paper describes the development and experimental results of navigation algorithms for an autonomous underwater glider (AUG) that employs an on-board acoustic Doppler current profiler (ADCP). AUGs are buoyancy-driven autonomous underwater vehicles that uses small hydrofoils to make forward progress while profiling vertically. During each dive, which can last up to 6 hours, the Seaglider AUG used in this experiment typically reaches the depth of 1,000 m and travels 3-6 km horizontally through the water, relying solely on dead-reckoning. Horizontal through-the-water (TTW) progress of AUG is 20-30 cm/s, which is comparable to the speed of the stronger ocean currents. Underwater navigation of an AUG in the presence of unknown advection therefore presents a considerable challenge. We present a post-processing algorithm to simultaneously estimate the ocean current profile through which the AUG was flown, as well as the AUG's horizontal position as influenced by the local currents. Extensive work by Todd et al in this domain employed a modified version of the lowered ADCP estimation framework \cite{Todd2017,Todd2011}.  In addition to this estimation framework, our work uses current data from the entire dive (as opposed to just during descent) and incorporates nonlinear hydrodynamic and attitude models from the vehicle navigation system to produce a time-varying current profile for the entire dive.

Results are demonstrated using ADCP data collected from two Seaglider AUGs deployed for 49 days off the north coast of Alaska during August and September 2017. Each glider was equipped with a Nortek Signature1000 1 MHz ADCP. Figure \ref{fig.SG} shows an ADCP installed, facing upwards, in the tail section of one of the AUGs. The ADCPs are configured by Nortek to use only three beams---in the upward-facing position, those are the two side beams and the aft beam during descent, and the two side beams and the forward beam during ascent. The ADCPs were programmed to collect a profile every 15 seconds. Each profile covers up to 20 m depth, with a new profile starting approximately every 1.5 m (i.e., the vertical distance covered by the AUG in 15 seconds).

\begin{figure}[!ht]
  \centering
  \includegraphics[width=3.0in]{./figs/Gliders_hires-99_crop.jpg}
  %\vspace{-0.1in}
  \caption{Seaglider AUG with an upwards-facing ADCP installed in aft fairing.}
  \label{fig.SG}
%   \vspace{-0.1in}
  %\rule{\textwidth}{0.02in}
  \vspace{-0.2in}
\end{figure}

The main challenge of AUG-based ADCP velocity profiling is that each velocity profile is measured relative to the TTW motion of the glider, which is generally unknown. TTW velocity can be inferred from the AUG dynamic model, but is subject to uncertainty because the model is based on steady state flight and doesn't take roll into account. This lack of a georeferenced platform velocity produces the need to perform a simultaneous robust estimation of current and glider TTW velocity using all available data -- ADCP velocity profiles, surrounding water density, glider buoyancy engine state, glider attitude and angle of attack, glider depth, and GPS positions at the start and end of the dive. Figure 2 shows a short section of the dive with the individual ADCP profiles, offset by our initial estimate of the AUG's TTW velocity (i.e., the velocity from the hydrodynamic model, without accounting for current-induced motion). The thick black line is the estimated current profile. This is from a 750 m dive, so the sections of data shown were collected 3.5 hrs apart. The difference in the current profile during ascent and descent are clearly visible, motivating the need for different current profiles during ascent and descent.

\begin{figure}[!ht]
  \centering
  \includegraphics[width=\columnwidth]{./figs/current_profile_snip.png}
  \vspace{-0.1in}
  \caption{ Overlapped ADCP profiles during descent (blue) and ascent (red) for dive 116 of sg198. The fitted current profile is shown as the thick black line.}
  \label{fig.profile_snip}
   \vspace{-0.1in}
  %\rule{\textwidth}{0.02in}
%   \vspace{-0.2in}
\end{figure}

\begin{thebibliography}{1}

\bibitem{Todd2017}
Todd, R.E., D.L. Rudnick, J.T. Sherman, W.B. Owens, L. George, Absolute velocity estimates from autonomous underwater gliders equipped with Doppler current profilers, \emph{J. Atmos. Oceanic Technol.}, 34(2), 309–333.%, doi:10.1175/JTECH-D-16-0156.1.

\bibitem{Todd2011}
Todd, R.E., D.L. Rudnick, M.R. Mazloff, R.E. Davis, B.D. Cornuelle, Poleward flows in the southern California Current System: Glider observations and numerical simulation, \emph{J. Geophys. Res.}, 116, C02026. %, doi:10.1029/2010JC006536.

\end{thebibliography}

\begin{abstract}
    
This paper describes the development and experimental results of navigation algorithms for an autonomous underwater glider (AUG) that employs an on-board acoustic Doppler current profiler (ADCP). AUGs are buoyancy-driven autonomous underwater vehicles that use small hydrofoils to make forward progress while profiling vertically.
%
During each dive, which can last up to 6 hours, the Seaglider AUG used in this experiment typically reaches the depth of 1000 m and travels 3-6 km horizontally through the water, relying solely on dead-reckoning. Horizontal through-the-water (TTW) progress of AUG is 20-30 cm/s, which is comparable to the speed of the stronger ocean currents. Underwater navigation of an AUG in the presence of unknown advection therefore presents a considerable challenge. We present a post-processing algorithm to simultaneously estimate the ocean current profile through which the AUG was flown, as well as the AUG's horizontal position as influenced by the local currents. 
%
Results are demonstrated using 1 MHz ADCP data collected from two Seaglider AUGs deployed for 49 days off the north coast of Alaska during August and September 2017. ... say soemthing quantitative if possible ...

\end{abstract}


% TO DO
% Add Lora as an author
% 
%

\section{Introduction}

The goal of this work is to simultaneously estimate a) absolute (Earth-referenced) ocean velocity profile, and b) the absolute autonomous underwater glider (AUG) path over the bottom, based on on-board acoustic Doppler current profiler (ADCP) observations of relative current velocity profiles. 

The main challenge of AUG-based ADCP velocity profiling is that each velocity profile is measured relative to the through-the-water (TTW) motion of the glider, as opposed to the earth referenced glider velocity. The AUG's TTW velocity can be inferred from the AUG dynamic model, but is subject to uncertainty because the model is based on steady state flight and doesn't take roll into account. This lack of a georeferenced platform velocity produces the need to perform a simultaneous robust estimation of current and glider TTW velocity using all available data -- ADCP velocity profiles, surrounding water density, glider buoyancy engine state, glider attitude and angle of attack, glider depth, and GPS positions at the start and end of the dive.

In this paper we will describe two different frameworks for performing this estimation. A linear global inverse, and a fast, robust nonlinear method. For comparison purposes in this initial work, we will use a pre-computed estimate of the glider TTW velocity from the hydrodynamic model for both methods. The goal, however, is to also incorporate the nonlinear hydrodynamic model estimation as part of the nonlinear estimation framework. We will also investigate the effect of including either a subsea position fix from range measurements during a small portion of the glider's flight.

\section{Background}

%Extensive work by Todd et al in this domain employed a modified version of the lowered ADCP estimation framework \cite{todd-2017-JAOT,todd-2011-JGRC}.  In addition to this estimation framework, our work uses current data from the entire dive (as opposed to just during descent) and incorporates nonlinear hydrodynamic and attitude models from the vehicle navigation system to produce a time-varying current profile for the entire dive.

Historically, ADCP measurements were made from ships with with relatively low frequency systems (70 kHz) or from moorings with low or mid-frequency instruments (300 kHz). The 300 kHz ADCPs are also common on large underwater vehicles, in particular instruments that can be used as a Doppler velocity log to provide ``bottom lock''---i.e., vehicle velocity relative to the seafloor \cite{yoerger-1998-ABE} or, in some specialized cases, the underside of ice \cite{mcfarland-2015-icerelnav}.

To provide better resolution of deep currents from shipboard measurements, lowered ADCP methods were developed using overlapping shear traces from a higher frequency (300 kHz) ADCP lowered from a (relatively) stationary ship. The shear traces are then used to reconstruct the full current profile \cite{Visbeck2002}.

In the last 10 years, there has been increased interest in using ADCPs from autonomous underwater vehicles to characterize the currents between the surface and the seafloor and to close the gap of uncertainty in vehicle drift between when GPS is available at the surface and bottom lock is obtained within range of the seafloor \cite{mstanway-2010a,medagoda10_oceans}.

Even more recently, very high frequency ADCPs (1 MHz) have been integrated onto underwater autonomous gliders---initially the Slocum gliders, and now onto Seagliders---and used to estimate depth-varying current profiles. 

\emph{ANDREY -- I think this is the reference we are missing \cite{thurnherr-2015-slocum-adcp}. \\
https://ieeexplore.ieee.org/document/7098134 \\
%https://ieeexplore.ieee.org/document/7098134
Can you summarize how it compares to your method? I admit the reviewer's comment sure sound like it could have been from Lou St. Laurent!}

Extensive work by Todd et al in this domain employed a modified version of the lowered ADCP estimation framework \cite{todd-2011-JGRC,todd-2017-JAOT}.
%
Our contribution builds on and extends their work with several key innovations. We use current data from the entire dive (as opposed to just during descent), which is crucial for multi-hour dives where, as we show, the current profiles on descent versus ascent differ significantly. We also separate the over-the-ground glider velocity into the drift velocity (as a result of advection), and the glider's horizontal through-the-water velocity. This enables us to directly incorporate hydrodynamic model velocity estimates and independently control the smoothness regularization of the different velocity components.
\section{Linear Inverse Method}

The goal of the inverse estimation is to simultaneously estimate a) absolute (Earth-referenced) ocean velocity profile, and b) the glider path through the water, based on on-board ADCP observations of relative current velocity profiles.
For simplicity, the derivation is in terms of scalar velocity $v$, which can be understood to be either one of the velocity components $(u,v)$, or a complex-number representation of velocity, $u+iv$.

\subsection {Formulation of the inverse problem}
Similar to \cite{Visbeck2002,todd-2011-JGRC}, glider-borne ADCP observations of horizontal currents are considered to be a sum of three unknown components: 
$$u^a(z,t)=u^o(z)-u^g(t)+u^n(z,t),$$
where 
\begin{itemize}
\item $u^o$ is the absolute ocean velocity 
\item $u^g$ is the over-the-ground (OTG) velocity of the glider platform 
\item $u^n$ is the ADCP measurement noise. 
\end{itemize}
Here, we separate the OTG glider velocity into the drift velocity $u^d=u^o(z^g(t))$, equal to the ocean velocity at the glider depth $z^g$, and the glider horizontal TTW "propulsion" speed $u^p$, which is determined by the glider control and its flight dynamics:
\[
u^g(t) = u^o(z^g(t)) + u^p(t).
\]
Thus the inverse problem is to obtain $u^o(z)$ and $u^g(t)$ such that the discrepancy with the observed values of $u^a$ are minimized,
\begin{equation}
\label{eq:min}
\min{\|u^a(z,t)-(u^o(z)-u^o(z^g(t))-u^p(t))\|_2},
\end{equation}
in this formulation in a least squares sense, but other minimization criteria can be used.

This formulation may appear unnecessarily complex compared to \cite{todd-2011-JGRC}, as it requires interpolation of ocean velocity onto the glider location ($u^o(z^g(t))$), but it allows explicit treatment of the glider control (discussed in Section \ref{sec:nonlin}). Additionally, we feel that application of the regularizing smoothness constraint (described below) to $u^p$ is more appropriate than to $u^g$.

\subsection{Discrete formulation}
We consider a set of individual ADCP observations $u^a_{ij}=u^a(z_{ij},t_j)$ at depth cells $z_{ij}$ and times $t_j$, with the
 $i\in [1,I]$ being the ADCP range cell index, and $j\in [1,J]$ being the sample time (ping) index.  All the valid observations form an observation column vector 
$$u^a=[u^a_1...u^a_k]^\top,$$
where the index $k\in [1, K], K\le IJ$ enumerates valid observations. The vectors $[z_k]$ and $[t_k]$ represent the corresponding depth and time coordinates of the $k$-th valid observation.


The unknown state vector 
$$x=\begin{bmatrix}
{u^g} \\ u^o
\end{bmatrix}$$
consists of a vector of the unknown glider OTG velocities 
$${u^g}= [u^g_1 ... u^g_M]^\top$$ 
defined on a temporal grid $\hat{t}_m, m\in[1,M]$, and a vector of the unknown ocean velocities 
$$u^o=[u^o_1 ... u^o_L]^\top$$
defined on a regular vertical grid $\hat{z}_l=(l-1)\Delta \hat{z}, l\in [1,L]$. For convenience, the temporal grid $\hat{t}_m$ is taken to be a superset of the sample times $t_j$, with the extra intervals added during the gaps in the ADCP record (at the beginning and the end of the dive, as well as during a brief intermission at the bottom of the dive); the spacing of the extra intervals is set to the average ADCP sampling interval.
Discrete formulation of the ADCP sampling problem \eqref{eq:min} then becomes 
$$
u^a= - H^t u^g + H^z u^0 + n, %\mathrm{Hx}+
$$
where $n$ is the noise vector.
%The observation matrix $H$ can be written as 
%$$
%H= \begin{bmatrix}
%&-H^t&|&H^{z} &\end{bmatrix},
%$$
The matrix operator $H^t$ represents temporal interpolation from the time grid $\{\hat{t}_m\}$ onto the sampling times $\{t_k\}$, and 
$H^{z}$ represents spatial interpolation from the vertical grid $\{\hat{z}_l\}$ onto $\{z_k\}$.
Since $\{t_k\}\subset \{\hat{t}_m\}$ by construction, $H^t$ is simply a subsampling matrix,
$$
H^t_{km}=\begin{cases} 1,&\text{if }t_k = \hat{t}_m\\0,&\text{otherwise } \end{cases}.
$$
The formulation of vertical interpolation matrices depends on the chosen interpolation method. Visbeck \cite{Visbeck2002} method is equivalent to the nearest-neighbor interpolation. Here, we employ linear interpolation, corresponding to a matrix operator
$$
H^z_{kl}=\begin{cases}
1-(\hat{z}_l-z_k)/\Delta \hat{z},&\hat{z}_{l-1} \le z_k <\hat{z}_l\\
1-(z_k-\hat{z}_l)/\Delta \hat{z},&\hat{z}_l \le z_k <\hat{z}_{l+1}\\
0,& \text{otherwise.}
\end{cases}
$$

\subsection {Additional constraints}
\label{sec:inverse.constraint}

At the core of the inverse problem is the requirement to balance various
constraints on the glider's motion in the form of additional measurements (such
as start- and end-of-dive GPS fixes) and assumptions we can make about the
physical environment, such as currents vary smoothly in time and space.


\textbf{Start- and End-of-Dive GPS:} GPS fixes before and after the dive provide tie-points on the integral $\int u^gdt$ over the duration of the dive. The discrete formulation of these relationships is given by 
$$
Sx\approx s,
$$
where $s$ is the scalar (complex) horizontal displacement between the two GPS fixes,
\(
S=\begin{bmatrix}w & 0\end{bmatrix},
\)
and $w$ represents the weights corresponding to trapezoidal rule integration, 
$$ w_m=\begin{cases}
0.5(\hat{t}_2-\hat{t}_1), &m=1\\
0.5(\hat{t}_{m+1}-\hat{t}_{m-1}), &1<m<M\\
0.5(\hat{t}_M-\hat{t}_{M-1}), &m=M
\end{cases}
$$

\textbf{Glider Velocity:} A measure of the glider's TTW velocity, such as that provided by the glider hydrodynamic model, can be used to provide a constraint on $u^p$(t) in \eqref{eq:min}. Discrete formulation of this constraint is 
\[
u^p = -H^t u^g + H^{z0} u^o + n, 
\]
%$$\begin{bmatrix}
%&-H^t&|&H^{z0} &\end{bmatrix} x=u^{model},$$
where $u^{p}$ is the TTW velocity  predicted by the hydrodynamic model, and $H^{z0}$ represents spatial interpolation of ocean velocity profile onto the glider depth 
(so that  $H^{z0}  x$ is the glider drift speed); this matrix is constructed in the same way as $H^{z}$. The relative confidence, within the model, of these measurements compared to the ADCP measurements can be controlled by adjusting the relative magnitude of the noise vectors $n$ for the two measurements. As is demonstrated and discussed in the Results section, however, understanding the subtleties of how to weight the constraints is still a work in progress.

\textbf{Smoothness Regularization:}
\label{sec:inv.reg}
Additionally, we require both the ocean and the glider velocities to be smooth, which is equivalent to minimization of ${D_2u^g}$, and ${D_2u^o}$ where $D_2$ are the second derivative operators of the appropriate sizes,
$$D_2=\begin{bmatrix}
1 &-2 &1 &0 &\cdots &0\\
0 &1 &-2 &1 &\\
\vdots & & & \ddots &\ddots\\
0 &\cdots &0 &1 &-2 &1
\end{bmatrix}.
$$


\textbf{Subtleties of Regularization:}
%
Ideally, we want to solve an inverse problem that is aware of both the
vehicle dynamics and control and the ADCP observations, because this is the
only way to make a self-consistent estimate. The challenge, as mentioned above
with the glider velocity estimates in particular, is that measurements are inherently
noisy, often biased, and optimally choosing the relative weighting between
measurements is still a work in progress.
%
%We have some idea how  the glider moves, and that knowledge should be useful.
%

In the extreme case, if we had a perfect hydrodynamic model, we would not need
the inverse at all, as we would know the TTW velocity at any moment (based
solely on the buoyancy and pitch control), and therefore would have absolute
ocean velocity estimates from every trace. At the other extreme, if the ADCP
measurements were perfect and extended close to the vehicle, we would have a
direct measure of the TTW velocity from that information alone and would not
need the dynamic model.

In reality, the ADCP is noisy and has gaps (notably, the blanking
distance). And without the hydrodynamic model, we can't distinguish between the legitimate accelerations due to AUG control (which
should be kept) and spurious accelerations arising from ADCP noise (which should
be eliminated)---and we would be forced to either keep both, or eliminate both,
which is suboptimal.

%so some smoothing and extrapolation is required to estimate the TTW
%velocity from ADCP. Smoothing things uniformly, however, presents another
%problem---we can expect the glider to move smoothly \emph{except} when it changes
%pitch or buoyancy.

%Including the model TTW velocity estimate constraint in the inverse should help, theoretically, as long as a) it is reasonably correct, and b) we have a good idea of what is considered "allowable" deviation from it -- because we need to tell the inverse what is considered to be "close" to the model TTW velocity (i.e., misfit measure). 

%Right now, KF has both the long-term (bias) and short-term (oversmoothing) issues, which are difficult to control simultaneously. (If it were only noisy, we could require the solution to be close to KF on average; if it were only biased, we could require the deviation to be smooth. Cannot efficiently require both.) 

%%%%%%%%%%%%%%

\subsection {Least-Squares Formulation}
The final least-squares problem formulation is given by 
$$
\min_x{\|{Gx-d}\|_2^2},
$$
where
$$
G=
\begin{bmatrix}
-H^t  &H^z \\
w & 0\\
 -H^t & H^{z0}\\
r_o D_2 & 0 \\
0  & r_gD_2
\end{bmatrix}, \qquad 
d=\begin{bmatrix}{u^a}\\ s\\  u^p \\ 0 \\ 0  \end{bmatrix},
$$
and $r_o$ and $r_g$ are regularization parameters.
The solution is given by
$$
x=(G^\top G)^{-1}G^\top d,
$$
and can be computed in a more efficient manner (e.g., using a QR decomposition aware of the block structure of $G$).
\subsection {Modification: two-profile solution}
The ocean velocity profile is expected to change over the duration of the dive. Therefore, it may be reasonable to seek \emph{two} ocean velocity profiles $u^d$ and $u^u$, corresponding to the descent and asent. The state vector and the observation matrix then become
$$x=\begin{bmatrix}
{u^g} \\ 
u^d \\
u^u 
\end{bmatrix},$$
$$
H= \begin{bmatrix}
&-H^t&|&H^{zd} &|&H^{zu} &\end{bmatrix},
$$
where the interpolation matrices $H^{zd}$ and $H^{zu}$ are constructed as $H^{z}$ before, except that only those rows of $H^{zd}$ that correspond to the downcast $z_k$ are non-zero, and vise versa. 

An additional constraint is necessary, requiring the ocean velocity profiles to match at the bottom end, i.e.
$$
\begin{bmatrix}
0 & e_n^\top & -e_n^\top
\end{bmatrix}x=0.
$$
where $e_n$ is the elementary vector with $1$ in the last entry. 

The modified system is given by 

$$
\min_x{\|{Gx-d}\|_2^2},
$$
where
$$
G=
\begin{bmatrix}
-H^t  &H^{zd} & H^{zu} \\
w & 0 & 0\\
 -H^t & H^{z0d} &H^{z0u} \\
r_o D_2 & 0 & 0\\
0  & r_gD_2 & r_gD_2 
\end{bmatrix}, \qquad 
d=\begin{bmatrix}{u^a}\\ s\\  u^p \\ 0 \\ 0  \end{bmatrix},
$$



For shorter dives, assuming identical current profiles on ascent and descent may be a reasonable assumptions. For the data collected over multi-hour dives, however, as we show, there can be significant deviation in the current profile during descent and ascent, and we use the two-profile solution.

\section{Nonlinear Method \textbf(SASHA)}

Sasha's method (w/o the hydrodynamic model piece for now)

Is this the right section title?

\section{Experimental Data Collection}
 
%Each glider was equipped with a Nortek Signature1000 1 MHz ADCP. Figure \ref{fig.SG} shows an ADCP installed, facing upwards, in the tail section of one of the AUGs. The ADCPs are configured by Nortek to use only three beams---in the upward-facing position, those are the two side beams and the aft beam during descent, and the two side beams and the forward beam during ascent. The ADCPs were programmed to collect a profile every 15 seconds. Each profile covers up to 20 m depth, with a new profile starting approximately every 1.5 m (i.e., the vertical distance covered by the AUG in 15 seconds).
 
 As part of the Canada Basin Glider Experiment (CABAGE), two Seagliders, SG196 and SG198, were deployed on 6 August 2017 at the shelf break north of Prudhoe Bay, AK. From there, they flew up to and around the CANAPE mooring array until they were recovered on 17 September 2017, for a total of 49 days.  Together the gliders covered approximately 1730 km over the course of 712 dives, with SG196 diving to 480 m depth and SG198 diving to 750 m.  Figure \ref{fig:track} shows the glider tracklines for both a short test deployment in 2016 and the 2017 deployment.

\begin{figure}%[!ht]
  \centering
  \includegraphics[width=0.9\columnwidth]{./figs/GliderTrajCombined.png}%{./figs/CABAGE_sg198_track_quicklook.png}
  \caption{SG196 and SG198 were deployed at the shelf break north of Prudhoe Bay, AK.  From there, they flew up to and around the CANAPE mooring array until, 49 days later, they were recovered by the USCGC Healy.}
  \label{fig:track}
\end{figure}

Each glider was equipped with a Nortek Signature1000 1 MHz ADCP, as well as the standard suite of conductivity temperature (CT) sensor, pressure sensor, WHOI MicroModem, and custom-built passive marine acoustic recorders (PMARs). Figure \ref{fig:SG} shows the gliders loaded on the R/V Ukpik, ready for launch, with the upward-facing ADCPs, installed in the tail section, clearly visible.

\begin{figure}%[!ht]
  \centering
%  \includegraphics[width=0.6\columnwidth]{./figs/Gliders_hires-99_crop.jpg}
%  \vspace{0.2cm}
  \includegraphics[width=0.9\columnwidth]{./figs/UkpikGliders.jpg}
  \caption{Ready for launch, SG196 and SG198 are loaded on the R/V Ukpik in Prudhoe Bay, AK, with the upward-facing ADCPs, installed in the aft fairing, clearly visible. }
  \label{fig:SG}
  \vspace{-0.2in}
\end{figure}

\begin{figure}%[!ht]
  \centering
  \includegraphics[width=0.9\columnwidth]{./figs/CABAGE_sg198_ADCP_quicklook.png}
  \caption{An example of the cumulative ADCP data collected by SG198 during the first few hundred kilometers of the deployment.}
  \label{fig:ADCP}
\end{figure}

For clarity during the discussion here, when referring to the ADCP data, we will use \emph{trace} to refer to an individual profile collected by the ADCP, and \emph{profile} to refer to the result of the inverse, i.e. the current profile for the entire dive. The ADCPs were programmed to collect a trace every 15 seconds with 2.0 m bins. Each trace typically covered 25m depth, with the actual usable range varying with the amount of acoustic scatterers in the water. As a result, with the typical vertical descent rate of 10 cm/s, any
given depth bin was covered by 15-16 different traces.  Figure \ref{fig:ADCP} illustrates the cumulative ADCP data collected by SG198 during the first few hundred kilometers of the 49-day deployment.
 
%\begin{figure}[!ht]
%  \centering
%  \includegraphics[width=3.0in]{./figs/Gliders_hires-99_crop.jpg}
%  %\vspace{-0.1in}
%  \caption{Seaglider AUG with an upwards-facing ADCP installed in aft %fairing.}
%  \label{fig.SG}
%%   \vspace{-0.1in}
%  %\rule{\textwidth}{0.02in}
%  \vspace{-0.2in}
%\end{figure}

\section{Results \textbf{SASHA}}

Figure \ref{fig.snippet} shows a short section of the current profile with the individual overlapping traces that have been aligned and averaged to produce the final current profile (black).  This is from a 750 m dive, so the sections of data shown were collected 3.5 hrs apart. The difference in the current profile during ascent and descent are clearly visible.

\begin{figure}%[!ht]
%  \includegraphics[width=\columnwidth]{./figs/current_profile_snip.png}
  \includegraphics[width=\columnwidth]{./figs/freeprofile_zoom_dv116.png}
  \caption{Overlapping ADCP traces during descent (blue) and ascent (red) for
    dive 99 of sg198, after alignment. The current profile produced by the linear inverse is shown as the thick black line.}
  \label{fig.snippet}
\end{figure}


\begin{figure}
  \includegraphics[width=\columnwidth]{./figs/downfull.pdf}
  \includegraphics[width=\columnwidth]{./figs/upfull.pdf}
  \caption{The above plot shows a comparison of the results from the linear method compared to the nonlinear method...}
  \label{fig.comparison}
\end{figure}

\begin{figure}%[!ht]
%  \centering
  \includegraphics[width=\columnwidth]{./figs/down100.pdf}
  \includegraphics[width=\columnwidth]{./figs/up100.pdf}
  \caption{The above plot shows a comparison between the the linear, nonlinear, and ground truth as obtained from a 600 kHz ADCP on the T3 mooring, which was 13 km away from the glider during this dive.}
  \label{fig.mooring}
\end{figure}






% \begin{figure}[!ht]
%   \centering
%   \includegraphics[width=\columnwidth]{./figs/current_profile_snip.png}
%   \vspace{-0.1in}
%   \caption{ Overlapped ADCP profiles during descent (blue) and ascent (red) for dive 116 of sg198. The fitted current profile is shown as the thick black line.}
%   \label{fig.profile_snip}
%   \vspace{-0.1in}
%   %\rule{\textwidth}{0.02in}
% %   \vspace{-0.2in}
% \end{figure}


\section{Conclusions and Future Work \textbf(SASHA)}

- all of the things that we will address

- using a single subsea position fix or multiple subsea range measurements to further constrain the output (I think only the nonlinear model can do this gracefully / easily?)


\section*{Acknowledgements \textbf(SARAH)}
We would like to acknowledge...Ukpik crew...Jason and Craig and Geoff and Ben and Wendy...ONR funding...DRDC funding...Peter and Matt.



\bibliographystyle{IEEEtran}
\bibliography{sew_additional}

% that's all folks
\end{document}